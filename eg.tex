%----------------------------------------------------------------
% lijin's homework template
% Modified by lijin, 11 Mar 2018
% Use this file as input in main file: %----------------------------------------------------------------
% lijin's homework template
% Modified by lijin, 11 Mar 2018
% Use this file as input in main file: %----------------------------------------------------------------
% lijin's homework template
% Modified by lijin, 11 Mar 2018
% Use this file as input in main file: \input{ljhwh.tex}
% Compile with cmdlet: xelatex -shell-escape eg.tex
% -------Original Version---------------
% HMC Math dept HW template example
% v0.04 by Eric J. Malm, 10 Mar 2005
%----------------------------------------------------------------
\documentclass[12pt,letterpaper,boxed]{hmcpset}

% set 1-inch margins in the document
\usepackage[margin=1in]{geometry}

% include this if you want to import graphics files with /includegraphics
\usepackage{graphicx}

% include this to insert MATLAB code, this will also load package xcolor
\usepackage[framed,numbered]{matlab-prettifier}

%-----------
% Define Colors
%-----------
%\usepackage[usenames,dvipsnames]{xcolor}
\definecolor{LightGray}{gray}{0.9}

%----------
% 添加缩进为两个元素
%----------
\usepackage{indentfirst}
\setlength{\parindent}{2em}

%----------
% 定义中文环境
%----------
\usepackage{xeCJK}
\setCJKmainfont[BoldFont={SimHei},ItalicFont={KaiTi}]{SimSun}
\setCJKsansfont{SimHei}
\setCJKfamilyfont{zhsong}{SimSun}
\setCJKfamilyfont{zhhei}{SimHei}
\setCJKfamilyfont{zhkai}{KaiTi}
\setCJKfamilyfont{zhfs}{FangSong}
\setCJKfamilyfont{zhli}{LiSu}
\setCJKfamilyfont{zhyou}{YouYuan}
\newcommand*{\songti}{\CJKfamily{zhsong}} % 宋体
\newcommand*{\heiti}{\CJKfamily{zhhei}}   % 黑体
\newcommand*{\kaiti}{\CJKfamily{zhkai}}   % 楷体
\newcommand*{\fangsong}{\CJKfamily{zhfs}} % 仿宋
\newcommand*{\lishu}{\CJKfamily{zhli}}    % 隶书
\newcommand*{\yuanti}{\CJKfamily{zhyou}}  % 圆体

% No new page when make title
\newcommand{\inlinemaketitle}{{\let\newpage\relax\maketitle}}

%-----------
% 输入化学式
%-----------
\usepackage[version=4]{mhchem}

%-----------
% 代码排版
%-----------
% 若重定义myminted,则无需包含minted
%\usepackage[cache=false]{minted}
\usepackage[most, minted]{tcolorbox}
\newtcblisting{myminted}[2][]{%
    listing engine=minted,
    minted language=#2,
    listing only,
    breakable,
    enhanced,
    top=0mm,
    bottom=0mm,
    opacityframe=0,
    borderline west={2pt}{-1pt}{black},
    sharp corners,
    %attach boxed title to top center = {yshift = -\tcboxedtitleheight/2},
    minted options = {
        linenos, 
        breaklines=true, 
        %breakbefore=., 
        fontsize=\footnotesize, 
        numbersep=2mm
    },
    overlay={%
        \begin{tcbclipinterior}
            %\fill[gray!25] (frame.south west) rectangle ([xshift=4mm]frame.north west);
            \fill[gray!25] ([yshift=0.5mm]frame.south west) rectangle ([yshift=-0.5mm,xshift=4mm]frame.north west);
        \end{tcbclipinterior}
    },
    #1   
}

% info for header block in upper right hand corner
\class{2014011716 工物41 李锦}
\assignment{第1章作业}
\duedate{\today}
%\name{Eric Malm}
%\class{MA 198}
%\assignment{Homework \#1}
%\duedate{09/03/2004}

% Compile with cmdlet: xelatex -shell-escape eg.tex
% -------Original Version---------------
% HMC Math dept HW template example
% v0.04 by Eric J. Malm, 10 Mar 2005
%----------------------------------------------------------------
\documentclass[12pt,letterpaper,boxed]{hmcpset}

% set 1-inch margins in the document
\usepackage[margin=1in]{geometry}

% include this if you want to import graphics files with /includegraphics
\usepackage{graphicx}

% include this to insert MATLAB code, this will also load package xcolor
\usepackage[framed,numbered]{matlab-prettifier}

%-----------
% Define Colors
%-----------
%\usepackage[usenames,dvipsnames]{xcolor}
\definecolor{LightGray}{gray}{0.9}

%----------
% 添加缩进为两个元素
%----------
\usepackage{indentfirst}
\setlength{\parindent}{2em}

%----------
% 定义中文环境
%----------
\usepackage{xeCJK}
\setCJKmainfont[BoldFont={SimHei},ItalicFont={KaiTi}]{SimSun}
\setCJKsansfont{SimHei}
\setCJKfamilyfont{zhsong}{SimSun}
\setCJKfamilyfont{zhhei}{SimHei}
\setCJKfamilyfont{zhkai}{KaiTi}
\setCJKfamilyfont{zhfs}{FangSong}
\setCJKfamilyfont{zhli}{LiSu}
\setCJKfamilyfont{zhyou}{YouYuan}
\newcommand*{\songti}{\CJKfamily{zhsong}} % 宋体
\newcommand*{\heiti}{\CJKfamily{zhhei}}   % 黑体
\newcommand*{\kaiti}{\CJKfamily{zhkai}}   % 楷体
\newcommand*{\fangsong}{\CJKfamily{zhfs}} % 仿宋
\newcommand*{\lishu}{\CJKfamily{zhli}}    % 隶书
\newcommand*{\yuanti}{\CJKfamily{zhyou}}  % 圆体

% No new page when make title
\newcommand{\inlinemaketitle}{{\let\newpage\relax\maketitle}}

%-----------
% 输入化学式
%-----------
\usepackage[version=4]{mhchem}

%-----------
% 代码排版
%-----------
% 若重定义myminted,则无需包含minted
%\usepackage[cache=false]{minted}
\usepackage[most, minted]{tcolorbox}
\newtcblisting{myminted}[2][]{%
    listing engine=minted,
    minted language=#2,
    listing only,
    breakable,
    enhanced,
    top=0mm,
    bottom=0mm,
    opacityframe=0,
    borderline west={2pt}{-1pt}{black},
    sharp corners,
    %attach boxed title to top center = {yshift = -\tcboxedtitleheight/2},
    minted options = {
        linenos, 
        breaklines=true, 
        %breakbefore=., 
        fontsize=\footnotesize, 
        numbersep=2mm
    },
    overlay={%
        \begin{tcbclipinterior}
            %\fill[gray!25] (frame.south west) rectangle ([xshift=4mm]frame.north west);
            \fill[gray!25] ([yshift=0.5mm]frame.south west) rectangle ([yshift=-0.5mm,xshift=4mm]frame.north west);
        \end{tcbclipinterior}
    },
    #1   
}

% info for header block in upper right hand corner
\class{2014011716 工物41 李锦}
\assignment{第1章作业}
\duedate{\today}
%\name{Eric Malm}
%\class{MA 198}
%\assignment{Homework \#1}
%\duedate{09/03/2004}

% Compile with cmdlet: xelatex -shell-escape eg.tex
% -------Original Version---------------
% HMC Math dept HW template example
% v0.04 by Eric J. Malm, 10 Mar 2005
%----------------------------------------------------------------
\documentclass[12pt,letterpaper,boxed]{hmcpset}

% set 1-inch margins in the document
\usepackage[margin=1in]{geometry}

% include this if you want to import graphics files with /includegraphics
\usepackage{graphicx}

% include this to insert MATLAB code, this will also load package xcolor
\usepackage[framed,numbered]{matlab-prettifier}

%-----------
% Define Colors
%-----------
%\usepackage[usenames,dvipsnames]{xcolor}
\definecolor{LightGray}{gray}{0.9}

%----------
% 添加缩进为两个元素
%----------
\usepackage{indentfirst}
\setlength{\parindent}{2em}

%----------
% 定义中文环境
%----------
\usepackage{xeCJK}
\setCJKmainfont[BoldFont={SimHei},ItalicFont={KaiTi}]{SimSun}
\setCJKsansfont{SimHei}
\setCJKfamilyfont{zhsong}{SimSun}
\setCJKfamilyfont{zhhei}{SimHei}
\setCJKfamilyfont{zhkai}{KaiTi}
\setCJKfamilyfont{zhfs}{FangSong}
\setCJKfamilyfont{zhli}{LiSu}
\setCJKfamilyfont{zhyou}{YouYuan}
\newcommand*{\songti}{\CJKfamily{zhsong}} % 宋体
\newcommand*{\heiti}{\CJKfamily{zhhei}}   % 黑体
\newcommand*{\kaiti}{\CJKfamily{zhkai}}   % 楷体
\newcommand*{\fangsong}{\CJKfamily{zhfs}} % 仿宋
\newcommand*{\lishu}{\CJKfamily{zhli}}    % 隶书
\newcommand*{\yuanti}{\CJKfamily{zhyou}}  % 圆体

% No new page when make title
\newcommand{\inlinemaketitle}{{\let\newpage\relax\maketitle}}

%-----------
% 输入化学式
%-----------
\usepackage[version=4]{mhchem}

%-----------
% 代码排版
%-----------
% 若重定义myminted,则无需包含minted
%\usepackage[cache=false]{minted}
\usepackage[most, minted]{tcolorbox}
\newtcblisting{myminted}[2][]{%
    listing engine=minted,
    minted language=#2,
    listing only,
    breakable,
    enhanced,
    top=0mm,
    bottom=0mm,
    opacityframe=0,
    borderline west={2pt}{-1pt}{black},
    sharp corners,
    %attach boxed title to top center = {yshift = -\tcboxedtitleheight/2},
    minted options = {
        linenos, 
        breaklines=true, 
        %breakbefore=., 
        fontsize=\footnotesize, 
        numbersep=2mm
    },
    overlay={%
        \begin{tcbclipinterior}
            %\fill[gray!25] (frame.south west) rectangle ([xshift=4mm]frame.north west);
            \fill[gray!25] ([yshift=0.5mm]frame.south west) rectangle ([yshift=-0.5mm,xshift=4mm]frame.north west);
        \end{tcbclipinterior}
    },
    #1   
}

% info for header block in upper right hand corner
\class{2014011716 工物41 李锦}
\assignment{第1章作业}
\duedate{\today}
%\name{Eric Malm}
%\class{MA 198}
%\assignment{Homework \#1}
%\duedate{09/03/2004}

\usepackage[most, minted]{tcolorbox}
\newtcblisting{myminted}[2][]{%
    listing engine=minted,
    minted language=#2,
    listing only,
    breakable,
    enhanced,
    top=0mm,
    bottom=0mm,
    opacityframe=0,
    borderline west={2pt}{-1pt}{black},
    sharp corners,
    %attach boxed title to top center = {yshift = -\tcboxedtitleheight/2},
    minted options = {
        linenos, 
        breaklines=true, 
        %breakbefore=., 
        fontsize=\footnotesize, 
        numbersep=2mm
    },
    overlay={%
        \begin{tcbclipinterior}
            %\fill[gray!25] (frame.south west) rectangle ([xshift=4mm]frame.north west);
            \fill[gray!25] ([yshift=0.5mm]frame.south west) rectangle ([yshift=-0.5mm,xshift=4mm]frame.north west);
        \end{tcbclipinterior}
    },
    #1   
}

%----------
% 插入latex代码及预览
%----------
\definecolor{lstbgcolor}{rgb}{0.9,0.9,0.9}
\lstloadlanguages{[LaTeX]TeX}
 
\newenvironment{latexample}[1][language={[LaTeX]TeX}]
{\lstset{backgroundcolor=\color{lstbgcolor},
    keywordstyle=\color[rgb]{0,0,1},
    commentstyle=\color[rgb]{0.133,0.545,0.133},
    stringstyle=\color[rgb]{0.627,0.126,0.941},
    breaklines=true,
    prebreak = \raisebox{0ex}[0ex][0ex]{\ensuremath{\hookleftarrow}},
    frame=single,
    language={[LaTeX]TeX},
    basicstyle=\footnotesize\ttfamily, #1}
  \VerbatimEnvironment\begin{VerbatimOut}{latexample.verb.out}}
  {\end{VerbatimOut}\noindent
  \\[-0.7em]
  \begin{minipage}{0.5\linewidth}
    \lstinputlisting[]{latexample.verb.out}%
  \end{minipage}\qquad
  \begin{minipage}{0.45\linewidth}
    \input{latexample.verb.out}
  \end{minipage}\\
}


%----------
% 插入latex代码及预览
%----------
%\usepackage{xcolor}
%\definecolor{lstbgcolor}{rgb}{0.9,0.9,0.9} 
% 
%\usepackage{listings}
%\lstloadlanguages{[LaTeX]TeX}
% 
%\usepackage{fancyvrb}
\definecolor{lstbgcolor}{rgb}{0.9,0.9,0.9}
\lstloadlanguages{[LaTeX]TeX}

\newenvironment{latexamplev}[1][language={[LaTeX]TeX}]
{\lstset{backgroundcolor=\color{lstbgcolor},
    keywordstyle=\color[rgb]{0,0,1},
    commentstyle=\color[rgb]{0.133,0.545,0.133},
    stringstyle=\color[rgb]{0.627,0.126,0.941},
    breaklines=true,
    prebreak = \raisebox{0ex}[0ex][0ex]{\ensuremath{\hookleftarrow}},
    frame=single,
    language={[LaTeX]TeX},
    basicstyle=\footnotesize\ttfamily, #1}
  \VerbatimEnvironment\begin{VerbatimOut}{latexamplev.verb.out}}
  {\end{VerbatimOut}\noindent
  \\[-0.7em]
  \begin{minipage}{\linewidth}
    \lstinputlisting[]{latexamplev.verb.out}%
  \end{minipage}\qquad \\
  \begin{minipage}{\linewidth}
    \input{latexamplev.verb.out}
  \end{minipage}\\
}



\name{lijin}
\class{LJhomework}
\assignment{Introduction and usage of this template}
\duedate{\today}

\begin{document}

\title{\vspace{-1cm} \textbf{Introduction \& Usage} \vspace{-0.5cm}}
\date{}
\inlinemaketitle


\textit{This is an example of the effect of this template:} \\
\vspace{1cm}

\begin{problem}[0]
某纯$\beta$放射性核素的测量结果如下表所示。请计算该核素的半衰期、衰变常数,并判断是那种核素。

    \begin{table}[H]
    \centering
    \label{表格}
    \begin{tabular}{|c|c|c|c|c|c|c|c|}
    \hline
    时间(天) & 0 & 1 & 2 & 3 & 5 & 10 & 20 \\
    \hline
    计数率(cpm) & 5500 & 5240 & 5000 & 4750 & 4320 & 3400 & 2050 \\
    \hline
    \end{tabular}
    \end{table}

\end{problem}

\begin{solution}
  \par
  放射性衰减规律为$N=N_0 e^{-\lambda t}$,用该公式对表中的数据进行拟合。使用MATLAB代码如下:
  \par
  %\begin{lstlisting}[style=Matlab-editor,basicstyle=\mlttfamily]
  %\begin{minted}[linenos,breaklines,breakanywhere,autogobble,frame=lines,bgcolor=LightGray,framesep=2mm,xleftmargin=2em]{matlab}
\begin{myminted}{matlab}
ti=[0 1 2 3 5 10 20]';
ni=[5500 5240 5000 4750 4320 3400 2050]';
syms t;
f=fittype('N*exp(-lambda*t)','independent','t','coefficients',{'N','lambda'});
fun=fit(ti,ni,f);
\end{myminted}
  %\end{minted}
  %\end{lstlisting}
  \par
  上述代码给出的拟合结果为:
	\[\begin{split}
		N_0     &  = 5507\ \ (5484, 5530) \quad \si{cpm} \\
		\lambda &  = 0.0489\ \ (0.04808, 0.04972) \quad \si{d^{-1}} \\
	\end{split}\]
  其中括号表示95\%的置信区间。
  \par
  因此,该核素的衰变常数为$0.0489 \text{d}^{-1}$,根据半衰期与衰变常数的关系:
	\[T_{1/2}=\frac{\text{ln}2}{\lambda}=14.2  \text{d}\]
  即该核素的半衰期为14.2天。查得此核素为$\sideset{^{32}_{15}}{}{\mathop{\mathrm{P}}}$。
\end{solution}

\newpage
\textit{Following is the introduction of some functions in this template:} \\
\vspace{1cm}

\problemlist{Pictures, Tables, Codes, Math, Enumerate, Circuits} 
\vspace{1cm}

\begin{problem}[1]
	\textbf{Insert Pictures}
	\begin{enumerate}[nosep,label=(\arabic*)]
    \item Single picture
		\item Pictures in rows and columns
		\item Wraped picture
	\end{enumerate}
\end{problem}

\begin{solution}
	\begin{enumerate}[nosep,label=(\arabic*)]
		\item \textbf{Single picture} \\
			\begin{itemize}
	        \item Using center environment to insert an picture: \\
\begin{latexample}[]
\begin{center}
  \includegraphics[width=0.5\textwidth]{./pic/latex.png}
\end{center}
\end{latexample}
		    \item Using figure environment to insert an picture with description: \\
\begin{latexample}[]
\begin{figure}[H]
  \centering
  \includegraphics[width=0.5\textwidth]{./pic/latex.png}
  \caption{eg}
\end{figure}
\end{latexample}
			\end{itemize}

		\item \textbf{Pictures in rows and columns}
			\begin{itemize}
				\item Multi-rows and multi-columns \\
\begin{latexamplev}[]
\begin{figure}[H]
  \captionsetup{name={Fig.},font={small}}
  \centering
  \begin{minipage}[b]{0.3\textwidth}
	\centering
	\includegraphics[width=0.7\textwidth]{./pic/01.png}
	\caption{eg1}
  \end{minipage}
  %\hspace{0.05\textwidth}
  \begin{minipage}[b]{0.3\textwidth}
	\centering
	\includegraphics[width=0.7\textwidth]{./pic/02.png}
	\caption{eg2}
  \end{minipage} \\
  \begin{minipage}[b]{0.3\textwidth}
	\centering
	\includegraphics[width=0.7\textwidth]{./pic/03.png}
	\caption{eg3}
  \end{minipage}
  \hspace{-0.03\textwidth}
  \begin{minipage}[b]{0.3\textwidth}
	\centering
	\includegraphics[width=0.7\textwidth]{./pic/04.png}
	\caption{eg4}
  \end{minipage}
  \hspace{-0.03\textwidth}
  \begin{minipage}[b]{0.3\textwidth}
	\centering
	\includegraphics[width=0.7\textwidth]{./pic/05.png}
	\caption{eg5}
  \end{minipage}
 \end{figure}
\end{latexamplev}

				\item Subfigures in columns: \\
\begin{latexamplev}[]
\begin{figure}[H]
  \centering
  \subfigure[first subfigure]{
	\begin{minipage}[b]{0.3\textwidth}
      \includegraphics[width=1\textwidth]{./pic/06.png} \\
	  \vspace{0.3em}
	  \includegraphics[width=1\textwidth]{./pic/latex.png}
	\end{minipage}
  }
  \hspace{2em}
  \subfigure[second subfigure]{
	\begin{minipage}[b]{0.3\textwidth}
      \centering
	  \includegraphics[angle=90,width=1.0\textwidth]{./pic/06.png}\vspace{1em} \\
	  \includegraphics[angle=90,width=0.7\textwidth]{./pic/latex.png}
	\end{minipage}
  }
  \caption{vertical subfigures}
\end{figure}
\end{latexamplev}
			%	\item Subfigures as one big figure
			\end{itemize}

		\item \textbf{Wraped picture} \\
			Insert a figure with texts around: \\
\begin{latexample}[]
\begin{wrapfigure}{r}{0pt}
  \includegraphics[width=2.5cm]{./pic/07.png}
\end{wrapfigure}
\renewcommand{\rubysize}{0.6}
\renewcommand{\rubysep}{0.2pt}
\textbf{Emoji} (Japanese: \ruby{絵}{え}\ruby{文}{も}\ruby{字}{じ}) are ideograms and smileys used in electronic messages and web pages. Emoji exist in various genres, including facial expressions, common objects, places and types of weather, and animals. They are much like emoticons, but emoji are actual pictures instead of typographics. 
\end{latexample}
	\end{enumerate}
\end{solution}

\begin{problem}[2]
	\textbf{Insert Tables}
	\begin{enumerate}[nosep,label=(\arabic*)]
		\item A simple table 
		\item Three-line table
		\item Table with slash header
	\end{enumerate}
\end{problem}

\begin{solution}
	\begin{enumerate}[nosep,label=(\arabic*)]

		\item \textbf{A simple table} \\
\begin{latexamplev}[]
\begin{table}[H]
  \centering
  \caption{Radiation decrease}
  \label{rad}
  \begin{tabular}{|c|c|c|c|c|c|c|c|}
	\hline
	时间(天) & 0 & 1 & 2 & 3 & 5 & 10 & 20 \\
	\hline
	计数率(cpm) & 5500 & 5240 & 5000 & 4750 & 4320 & 3400 & 2050 \\
	\hline
  \end{tabular}
\end{table}
\end{latexamplev}

		\item \textbf{Table with slash header} \\
\begin{latexamplev}[]
\begin{table}[H]
  \centering
  \begin{tabular}{c|cc}
	\hline
	\diagbox{$p$/\si{MPa}}{$t$/\si{\degreeCelsius}} & 300 & 350 \\
	\hline
	15.0 & 565.8 & \underline{112.8} \\
	17.5 & 570.5 & 452.5 \\
	20.0 & 575.5 & 465.0 \\
	\hline
  \end{tabular}
\end{table}
\end{latexamplev}

		\item \textbf{Three-line table} \\
\begin{latexamplev}[]
\begin{table}[H]
	\centering
	\begin{tabular}{ccc}
		\toprule
		温度$t$/\si{\degreeCelsius} & 压力$p$/\si{\mega\pascal} & 比焓$h$/(\si{kJ/kg}) \\
		\midrule
		340 & 14.608 & 1596 \\
		350 & 16.537 & 1672 \\
		\bottomrule
	\end{tabular}
\end{table}
\end{latexamplev}

	\end{enumerate}
\end{solution}

\begin{problem}[3]
	\textbf{Codes Display}
	\begin{enumerate}[nosep,label=(\arabic*)]
	\item Codes for different languages
	\item \LaTeX \ code and its output
	\end{enumerate}
\end{problem}

\begin{solution}
	\begin{enumerate}[nosep,label=(\arabic*)]
	\item \textbf{Codes for different languages} \\
		Use \textit{myminted} environment to display C++, MATLAB, Python, ... and many other kinds of languages. To find all the language supported, execute command ``\verb|pygmentize| \verb| -L lexers|'' in your terminal.\\
\begin{latexample}[]
\begin{myminted}{c++}
#include <iostream>
using namespace std;

int main()
{
  cout<<"Hello, World!"<<endl;
  return 0;
}
\end{myminted}
\end{latexample}

	\item \textbf{\LaTeX \ code and its output} \\
		There are two types: one is that display \LaTeX codes box on left side and output on right side, the other is that display output under \LaTeX codes box.
		\begin{itemize}
			\item \textbf{Left-code \& right-output} \cndash \textit{latexample} environment \\
				Use \verb+%----------
% 插入latex代码及预览
%----------
\definecolor{lstbgcolor}{rgb}{0.9,0.9,0.9}
\lstloadlanguages{[LaTeX]TeX}
 
\newenvironment{latexample}[1][language={[LaTeX]TeX}]
{\lstset{backgroundcolor=\color{lstbgcolor},
    keywordstyle=\color[rgb]{0,0,1},
    commentstyle=\color[rgb]{0.133,0.545,0.133},
    stringstyle=\color[rgb]{0.627,0.126,0.941},
    breaklines=true,
    prebreak = \raisebox{0ex}[0ex][0ex]{\ensuremath{\hookleftarrow}},
    frame=single,
    language={[LaTeX]TeX},
    basicstyle=\footnotesize\ttfamily, #1}
  \VerbatimEnvironment\begin{VerbatimOut}{latexample.verb.out}}
  {\end{VerbatimOut}\noindent
  \\[-0.7em]
  \begin{minipage}{0.5\linewidth}
    \lstinputlisting[]{latexample.verb.out}%
  \end{minipage}\qquad
  \begin{minipage}{0.45\linewidth}
    \input{latexample.verb.out}
  \end{minipage}\\
}

+ at the beginning of the tex file. Use as this:
\begin{verbatim}
\begin{latexample}
	  ... LaTeX code here ...
\end{latexample}
\end{verbatim}
				Following is an example\label{enum} and the effect: \\
\begin{latexample}[]
\begin{enumerate}[nosep,label=(\arabic*)]
  \item first
    \begin{itemize}
	  \item something
	  \item something else
	\end{itemize}
  \item second
\end{enumerate}
\end{latexample}

			\item \textbf{Up-code \& down-output} \cndash \textit{latexamplev} environment \\
				Use \verb+%----------
% 插入latex代码及预览
%----------
%\usepackage{xcolor}
%\definecolor{lstbgcolor}{rgb}{0.9,0.9,0.9} 
% 
%\usepackage{listings}
%\lstloadlanguages{[LaTeX]TeX}
% 
%\usepackage{fancyvrb}
\definecolor{lstbgcolor}{rgb}{0.9,0.9,0.9}
\lstloadlanguages{[LaTeX]TeX}

\newenvironment{latexamplev}[1][language={[LaTeX]TeX}]
{\lstset{backgroundcolor=\color{lstbgcolor},
    keywordstyle=\color[rgb]{0,0,1},
    commentstyle=\color[rgb]{0.133,0.545,0.133},
    stringstyle=\color[rgb]{0.627,0.126,0.941},
    breaklines=true,
    prebreak = \raisebox{0ex}[0ex][0ex]{\ensuremath{\hookleftarrow}},
    frame=single,
    language={[LaTeX]TeX},
    basicstyle=\footnotesize\ttfamily, #1}
  \VerbatimEnvironment\begin{VerbatimOut}{latexamplev.verb.out}}
  {\end{VerbatimOut}\noindent
  \\[-0.7em]
  \begin{minipage}{\linewidth}
    \lstinputlisting[]{latexamplev.verb.out}%
  \end{minipage}\qquad \\
  \begin{minipage}{\linewidth}
    \input{latexamplev.verb.out}
  \end{minipage}\\
}

+ at the beginning of the tex file. Use as this:
\begin{verbatim}
\begin{latexamplev}
	  ... LaTeX code here ...
\end{latexamplev}
\end{verbatim}
				Following is an example and the effect: \\
\begin{latexamplev}[]
\begin{enumerate}[nosep,label=(\arabic*)]
  \item first
    \begin{itemize}
	  \item something
	  \item something else
	\end{itemize}
  \item second
\end{enumerate}
\end{latexamplev}

		\end{itemize}
	\end{enumerate}

\end{solution}

\begin{problem}[4]
	\textbf{Math Input}
\end{problem}

\begin{solution}
	\hypersetup{urlcolor=purple}
	This is a very huge question and I recommand to read a Chinese book \href{http://www.latexstudio.net/wp-content/uploads/2018/02/ChinaTeXMathFAQ_V1.1.pdf}{China\TeX \ Math FAQ} or an English book \href{https://www.springer.com/cn/book/9780387688527#otherversion=9780387322896}{More Math Into \LaTeX}.	
\end{solution}

\begin{problem}[5]
	\textbf{Enumerate \& Itemize}
\end{problem}

\begin{solution}
	%\hypersetup{citecolor=purple}
	As shows the example of Problem 3, \href{enum}{\ref{enum}}. Here package \textit{enumitem} instead of package \textit{enumerate} \ is used for enumerate and package \textit{itemize}\ is used for itemize. \\
\begin{latexample}[]
\begin{enumerate}[nosep,label=(\arabic*)]
  \item first
	  \begin{itemize}
	  \item something
	  \item something else
	\end{itemize}
  \item second
\end{enumerate}
\end{latexample}

\end{solution}

\begin{problem}[6]
	\textbf{Plot Circuits}
\end{problem}

\begin{solution}
	Both package \textit{circuitikz} \ and\ \textit{circ} can be used for plotting circuits. \textit{circuitikz} \ is chosen in this template. 
	\hypersetup{urlcolor=purple}
	\par Use command ``\verb|texdoc circuitikz|'' in your terminal to see its manual. And here is a \href{http://topspeedsnail.com/latex-circuitikz-circuit}{webpage} for introducing how to use it.
	\par Here is an example: \\
\begin{latexamplev}[]
\begin{figure}[H]
  \centering
  \begin{circuitikz}
    \draw (0,0) node [anchor=east] {\ctikzlabel{$\ U_0$}{$12\si{V}$}} to [R,l=$R_A$,o-*]
  	(3,0) to[short] (3,1) to [R,l=$R_B$] (7,1)
  	to[short] (7,0) to [R,l=$R_C$,*-o] (10,0) node [anchor=west] {\ctikzlabel{$\ U_1$}{$15\si{V}$}};
    \draw (3,0) to[short] (3,-1) to [R,l=$R_D$] (7,-1) to[short] (7,0);
  \end{circuitikz}
\end{figure}
\end{latexamplev}
\end{solution}

\end{document}
